\usepackage[utf8]{inputenc}
\usepackage[T1]{fontenc} % Zeichensatzkodierung

\usepackage{calc} % Berechnungen

\usepackage[english]{babel} % Deutsche Lokalisierung
\usepackage{graphicx} % Grafiken
\usepackage[absolute, overlay]{textpos} % Positionierung

\usepackage[symbol]{footmisc}
\usepackage{url}
\usepackage[pages=some]{background}
\usepackage{framed}
\usepackage{amsmath}

\backgroundsetup{
	scale=1,
	%color=black,
	%opacity=0.4,
	angle=0,
	contents={%
		\includegraphics[height=\paperheight]{questions.pdf}
	}%
}

% Silbentrennung:
\usepackage{hyphenat}
%\tolerance 2414
%\hbadness 2414
%\emergencystretch 1.5em
%\hfuzz 0.3pt
%\widowpenalty=10000     % Hurenkinder
%\clubpenalty=10000      % Schusterjungen
%\vfuzz \hfuzz

% Euro-Symbol:
\usepackage[gen]{eurosym}
\DeclareUnicodeCharacter{20AC}{\euro{}}

% Schriftart Helvetica:
\usepackage[scaled]{helvet}
\renewcommand{\familydefault}{\sfdefault}

\usepackage{mathptmx} % skalierbare Formelschriften

\usepackage{tabularx}

\usepackage{multicol} % mehrspaltiger Text

\usepackage{tikz}
\usetikzlibrary{arrows, shapes, shapes.multipart, trees, positioning,
    backgrounds, fit, matrix}

% Diagramme:
\usepackage{pgfplots}
\pgfplotsset{compat=default}

% Erweiterbare Fusszeile:
\newcommand{\PraesentationFusszeileZusatz}{}

\usefonttheme[onlymath]{serif}
\setbeamertemplate{bibliography item}[text]
\setbeamercolor{bibliography entry author}{fg=black}
\setbeamercolor{bibliography entry item}{fg=black}
\setbeamercolor{bibliography entry title}{fg=black} 
\setbeamercolor{bibliography entry location}{fg=black} 
\setbeamercolor{bibliography entry note}{fg=black}  

\makeatletter
\renewcommand*\env@matrix[1][\arraystretch]{%
  \edef\arraystretch{#1}%
  \hskip -\arraycolsep
  \let\@ifnextchar\new@ifnextchar
  \array{*\c@MaxMatrixCols c}}
\makeatother

